\documentclass[10pt]{article}
\usepackage{graphicx}
\def\inputGnumericTable{}
\usepackage[latin1]{inputenc}
\usepackage{fullpage}
\usepackage{color}
\usepackage{array}
\usepackage{longtable}
\usepackage{calc}
\usepackage{multirow}
\usepackage{hhline}
\usepackage{ifthen}
\usepackage[none]{hyphenat}
\usepackage{graphicx}
\usepackage{listings}
\usepackage[english]{babel}
\usepackage{siunitx}
\usepackage{graphicx}
\usepackage{caption} 
\usepackage{booktabs}
\usepackage{array}
\usepackage{gensymb}
\usepackage{amssymb} % for \because
\usepackage{amsmath}   % for having text in math mode
\usepackage{extarrows} % for Row operations arrows
\usepackage{listings}
%\usepackage[utf8]{inputenc}
\lstset{
  frame=single,
  breaklines=true
}
\usepackage{hyperref}
\usepackage[margin=0.5in]{geometry}
  
%Following 2 lines were added to remove the blank page at the beginning
\usepackage{atbegshi}% http://ctan.org/pkg/atbegshi
\AtBeginDocument{\AtBeginShipoutNext{\AtBeginShipoutDiscard}}


%New macro definitions
\renewcommand{\labelenumi}{(\roman{enumi})}
\newcommand{\mydet}[1]{\ensuremath{\begin{vmatrix}#1\end{vmatrix}}}
\providecommand{\brak}[1]{\ensuremath{\left(#1\right)}}
\newcommand{\solution}{\noindent \textbf{Solution: }}
\newcommand{\myvec}[1]{\ensuremath{\begin{pmatrix}#1\end{pmatrix}}}
\providecommand{\norm}[1]{\left\lVert#1\right\rVert}
\providecommand{\abs}[1]{\left\vert#1\right\vert}
\let\vec\mathbf{}
\begin{document}

\begin{center}
\title{\textbf{CIRCLES}}
\date{\vspace{-5ex}} %Not to print date automatically
\maketitle
\end{center}
\section{9$^{th}$ Maths - Chapter 10}

This is Problem 2 from Exercise-10.6\\\\
Two chords AB and CD of lengths 5 cm and 11 cm respectively of a circle are parallel
to each other and are on opposite sides of its centre. If the distance between AB and
CD is 6 cm, find the radius of the circle.
\section{construction}
\begin{figure}[h!]
	\begin{center}
		\includegraphics[width=5in]{./figs/fig.png}
	\end{center}
\caption{}
\label{fig:Fig1}
\end{figure}
The input parameters for this construction are\\
\begin{table}[h!]
	\centering
	\input{/sdcard/Download/codes/circles/9.10.6.2/table/table.tex}
\label{table:1}
\end{table}\\
\begin{align}
\vec{A}=r\myvec{\cos{\theta_1}\\\sin{\theta_1}},\vec{B}=r\myvec{\cos{\theta_2}\\\sin{\theta_2}},\vec{C}=r\myvec{\cos{\theta_3}\\\sin{\theta_3}},\vec{D}=r\myvec{\cos{\theta_4}\\\sin{\theta_4}}
\end{align}
\solution
Lines AB and CD are parallel.\\
Therefore,\\
\begin{align}
\vec{m_1}=&\vec{m_2}\\
\myvec{\cos{\theta_1}-\cos{\theta_2}\\\sin{\theta_1}-\sin{\theta_2}}=&\myvec{\cos{\theta_3}-\cos{\theta_4}\\\sin{\theta_3}-\sin{\theta_4}}\\
\implies\myvec{1\\-1}=&\myvec{1\\-1}
\label{eq:1}
\end{align}
from \eqref{eq:1} the normal vector is given by
\begin{align}
\vec{n}=\myvec{-1\\-1}
\end{align}
The line equation of AB is 
\begin{align}
r\myvec{-1&-1}\vec{x}=&r^2\myvec{-1&-1}\myvec{1\\0}\\
\myvec{-1&-1}=&-r
\label{eq:2}
\end{align}
The line equation of CD is 
\begin{align}
r\myvec{-1&-1}\vec{x}=&r^2\myvec{-1&-1}\myvec{0\\-1}\\
\myvec{-1&-1}=&r
\label{eq:3}
\end{align}
from \eqref{eq:2} and \eqref{eq:3}
\begin{align}
c_1=-r,c_2=r
\end{align}
The distance between parallel lines is
\begin{align}
d=&\frac{\abs{c_1-c_2}}{\norm{\vec{n}}}\\
\implies 6=&\frac{2r}{\sqrt{2}}\\
\implies r=&4.24
\end{align}
\end{document}
