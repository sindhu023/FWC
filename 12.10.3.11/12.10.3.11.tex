\documentclass[10pt]{article}
\usepackage{graphicx}
\usepackage[none]{hyphenat}
\usepackage{graphicx}
\usepackage{listings}
\usepackage[english]{babel}
\usepackage{siunitx}
\usepackage{graphicx}
\usepackage{caption} 
\usepackage{booktabs}
\usepackage{array}
\usepackage{amssymb} % for \because
\usepackage{amsmath}   % for having text in math mode
\usepackage{extarrows} % for Row operations arrows
\usepackage{listings}
\usepackage[utf8]{inputenc}
\lstset{
  frame=single,
  breaklines=true
}
\usepackage{hyperref}
  
%Following 2 lines were added to remove the blank page at the beginning
\usepackage{atbegshi}% http://ctan.org/pkg/atbegshi
\AtBeginDocument{\AtBeginShipoutNext{\AtBeginShipoutDiscard}}


%New macro definitions
\newcommand{\mydet}[1]{\ensuremath{\begin{vmatrix}#1\end{vmatrix}}}
\providecommand{\brak}[1]{\ensuremath{\left(#1\right)}}
\newcommand{\solution}{\noindent \textbf{Solution: }}
\newcommand{\myvec}[1]{\ensuremath{\begin{pmatrix}#1\end{pmatrix}}}
\providecommand{\norm}[1]{\left\lVert#1\right\rVert}
\providecommand{\abs}[1]{\left\vert#1\right\vert}
\let\vec\mathbf{}
\begin{document}

\begin{center}
\title{\textbf{Vector Algebra}}
\date{\vspace{-5ex}} %Not to print date automatically
\maketitle
\end{center}

\section{12$^{th}$ Maths - Chapter 10}
This is Problem 11 from Exercise-10.3
\begin{enumerate}
\item Show that $\abs{\overrightarrow{a}}\overrightarrow{b}+\abs{\overrightarrow{b}}\overrightarrow{a}$ is perpendicular to $\abs{\overrightarrow{a}}\overrightarrow{b}-\abs{\overrightarrow{b}}\overrightarrow{a}$, for any two nonzero vectors $\overrightarrow{a}$ and $\overrightarrow{b}$\\  

\solution
From the given information

\begin{align}
	\brak{\norm{\vec{a}}\vec{b}+\norm{\vec{b}}\vec{a}}^\top\brak{\norm{\vec{a}}\vec{b}-	\norm{\vec{b}}\vec{a}}\\
	\implies \norm{\vec{a}}^\top\vec{b}^\top\norm{\vec{a}}\vec{b}-\norm{\vec{a}}^\top\vec{b}^\top\norm{\vec{b}}\vec{a}+\norm{\vec{b}}^\top\vec{a}^\top\norm{\vec{a}}\vec{b}-\norm{\vec{b}}^\top\vec{a}^\top\norm{\vec{b}}\vec{a}
\end{align}
we know that
\begin{align}
    \vec{a}^{\top}\vec{a} = \norm{\vec{a}}^2
    \label{eq1}  \\
    \vec{b}^{\top}\vec{b} = \norm{\vec{b}}^2
    \label{eq2}  \\
    \vec{a}^{\top}\vec{b} = \vec{b}^{\top}\vec{a}
    \label{eq3}
\end{align}
By using \eqref{eq1} and \eqref{eq2} and \eqref{eq3}
\begin{align}
	\implies\norm{\vec{a}}^2\norm{\vec{b}}^2&-\norm{\vec{b}}^2\norm{\vec{a}}^2=0
\end{align}
\end{enumerate}
\end{document}