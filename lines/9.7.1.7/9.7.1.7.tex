\documentclass[10pt]{article}
\usepackage{graphicx}
\def\inputGnumericTable{}
\usepackage[latin1]{inputenc}
\usepackage{fullpage}
\usepackage{color}
\usepackage{array}
\usepackage{longtable}
\usepackage{calc}
\usepackage{multirow}
\usepackage{hhline}
\usepackage{ifthen}
\usepackage[none]{hyphenat}
\usepackage{graphicx}
\usepackage{listings}
\usepackage[english]{babel}
\usepackage{siunitx}
\usepackage{graphicx}
\usepackage{caption} 
\usepackage{booktabs}
\usepackage{array}
\usepackage{gensymb}
\usepackage{amssymb} % for \because
\usepackage{amsmath}   % for having text in math mode
\usepackage{extarrows} % for Row operations arrows
\usepackage{listings}
%\usepackage[utf8]{inputenc}
\lstset{
  frame=single,
  breaklines=true
}
\usepackage{hyperref}
\usepackage[margin=0.5in]{geometry}
  
%Following 2 lines were added to remove the blank page at the beginning
\usepackage{atbegshi}% http://ctan.org/pkg/atbegshi
\AtBeginDocument{\AtBeginShipoutNext{\AtBeginShipoutDiscard}}


%New macro definitions
\renewcommand{\labelenumi}{(\roman{enumi})}
\newcommand{\mydet}[1]{\ensuremath{\begin{vmatrix}#1\end{vmatrix}}}
\providecommand{\brak}[1]{\ensuremath{\left(#1\right)}}
\newcommand{\solution}{\noindent \textbf{Solution: }}
\newcommand{\myvec}[1]{\ensuremath{\begin{pmatrix}#1\end{pmatrix}}}
\providecommand{\norm}[1]{\left\lVert#1\right\rVert}
\providecommand{\abs}[1]{\left\vert#1\right\vert}
\let\vec\mathbf{}
\begin{document}

\begin{center}
\title{\textbf{TRIANGLES}}
\date{\vspace{-5ex}} %Not to print date automatically
\maketitle
\end{center}
\section{9$^{th}$ Maths - Chapter 7}

This is Problem 7 from Exercise-7.1\\\\
AB is a line segment and P is its mid-point.D and E are points on the same side of
AB such that $\angle$ BAD = $\angle$ ABE and $\angle$ EPA = $\angle$ DPB.Show that
\begin{enumerate}
\item $\triangle$ DAP $\cong$ $\triangle$ EBP
\item AD = BE
\end{enumerate}
\section{construction}
\begin{figure}[h!]
	\begin{center}
		\includegraphics[width=5in]{./figs/fig.png}
	\end{center}
\caption{}
\label{fig:Fig1}
\end{figure}
The input parameters for this construction are\\
\begin{table}[h!]
	\centering
 	\input{/sdcard/Download/codes/lines/9.7.1.7/table/table.tex}
\caption{}
\label{table}
\end{table}\\
\begin{center}
$\vec{A}=\myvec{-4\\0}$\\
$\vec{P}=\myvec{0\\0}$\\
$\vec{B}=\myvec{4\\0}$\\
$\vec{D}=5.43\myvec{\cos60\degree\\\sin60\degree}$\\
$\vec{E}=5.43\myvec{\cos120\degree\\\sin120\degree}$\\
\end{center}
\solution
Given\\
\begin{align}
AP = BP
\label{eq:1}\\
\angle BAD = \angle ABE
\label{eq:2}
\end{align}
\textbf{To Prove:}  $\triangle DAP \cong \triangle$ EPB\\
to prove $\angle {DPA}=\angle {EPB}$
\begin{align}
\text{ Let  }\theta2=&\angle DPA\\
\vec{m1}=&\vec{P}-\vec{D}=\myvec{-2.7\\-4.7}\\
\vec{m2}=&\vec{P}-\vec{A}=\myvec{4\\0}\\
\theta2=&\cos^{-1}\frac{\vec{m1}^\top\vec{m2}}{\norm{\vec{m1}}\norm{\vec{m2}}}\\
\implies\theta2=&\cos^{-1}\frac{\myvec{-2.7&-4.7}\myvec{4\\0}}{(5.42)(4)}=120\degree
\label{eq:3}\\
\theta3=&\angle EPB\\
\vec{n1}=&\vec{P}-\vec{E}=\myvec{2.7\\-4.7}\\
\vec{n2}=&\vec{P}-\vec{B}=\myvec{-4\\0}\\
\theta3 =& \cos^{-1}\frac{\vec{n1}^\top\vec{n2}}{\norm{\vec{n1}}\norm{\vec{n2}}}\\
\implies\theta3=&\cos^{-1}\frac{\myvec{2.7&-4.7}\myvec{-4\\0}}{(5.42)(4)}=120\degree
\label{eq:4}
\end{align}
from $\eqref{eq:3}$ and $\eqref{eq:4}$
\begin{align}
\angle {DPA}=&\angle {EPB}
\label{eq:5}
\end{align}
from $\eqref{eq:1},\eqref{eq:2}$ and $\eqref{eq:5}$\\
\begin{align}
\triangle DAP \cong \triangle EBP
\label{eq:6}
\end{align}
from $\eqref{eq:6}$\\
\begin{center}
AD=BE
\end{center}
\end{document}
