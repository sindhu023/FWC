\documentclass[journal,12pt,twocolumn]{IEEEtran}
%
\usepackage{setspace}
\usepackage{textcomp}
\usepackage{gensymb}
\usepackage{xcolor}
\usepackage{caption}
%\usepackage{subcaption}
%\doublespacing
\singlespacing

\usepackage{graphicx}
\graphicspath{{./images/}}
\usepackage[colorlinks=true, urlcolor=blue, linkcolor=black]{hyperref}
%\usepackage[parfill]{parskip}
%\usepackage{amssymb}
%\usepackage{relsize}
\usepackage[cmex10]{amsmath}
\usepackage{mathtools}
%\usepackage{amsthm}
%\interdisplaylinepenalty=2500
%\savesymbol{iint}
%\usepackage{txfonts}
%\restoresymbol{TXF}{iint}
%\usepackage{wasysym}
\usepackage{amsthm}
\usepackage{mathrsfs}
\usepackage{txfonts}
\usepackage{stfloats}
\usepackage{cite}
\usepackage{cases}
\usepackage{subfig}
%\usepackage{xtab}
\usepackage{hyperref}
\usepackage{longtable}
\usepackage{multirow}
%\usepackage{algorithm}
%\usepackage{algpseudocode}
\usepackage{enumitem}
\usepackage{mathtools}
%\usepackage{eenrc}
%\usepackage[framemethod=tikz]{mdframed}
%\usepackage{hyperref}
\usepackage{listings}
    \usepackage[latin1]{inputenc}                                 %%
    \usepackage{color}                                            %%
    \usepackage{array}                                            %%
    \usepackage{longtable}                                        %%
    \usepackage{calc}                                             %%
    \usepackage{multirow}                                         %%
    \usepackage{hhline}                                           %%
    \usepackage{ifthen}                                           %%
  %optionally (for landscape tables embedded in another document): %%
    \usepackage{lscape}     
\usepackage{tikz}
\usepackage{circuitikz}
\usepackage{karnaugh-map}
\usepackage{pgf}

\usepackage{url}
\def\UrlBreaks{\do\/\do-}



%\usepackage{stmaryrd}


%\usepackage{wasysym}
%\newcounter{MYtempeqncnt}
\DeclareMathOperator*{\Res}{Res}
%\renewcommand{\baselinestretch}{2}
\renewcommand\thesection{\arabic{section}}
\renewcommand\thesubsection{\thesection.\arabic{subsection}}
\renewcommand\thesubsubsection{\thesubsection.\arabic{subsubsection}}

\renewcommand\thesectiondis{\arabic{section}}
\renewcommand\thesubsectiondis{\thesectiondis.\arabic{subsection}}
\renewcommand\thesubsubsectiondis{\thesubsectiondis.\arabic{subsubsection}}



%\surroundwithmdframed[width=\columnwidth]{lstlisting}
\def\inputGnumericTable{}                                 %%
\lstset{
%language=C,
frame=single, 
breaklines=true,
columns=fullflexible
}
 

\begin{document}
%

\theoremstyle{definition}
\newtheorem{theorem}{Theorem}[section]
\newtheorem{problem}{Problem}
\newtheorem{proposition}{Proposition}[section]
\newtheorem{lemma}{Lemma}[section]
\newtheorem{corollary}[theorem]{Corollary}
\newtheorem{example}{Example}[section]
\newtheorem{definition}{Definition}[section]
%\newtheorem{algorithm}{Algorithm}[section]
%\newtheorem{cor}{Corollary}
\newcommand{\BEQA}{\begin{eqnarray}}
\newcommand{\EEQA}{\end{eqnarray}}
\newcommand{\define}{\stackrel{\triangle}{=}}
\vspace{2cm}
\title{ 
XOR logic through Arduino
}

\author{B.Sai Sindhu}


\maketitle
\tableofcontents
\bigskip
%
%\newpage
\section{Abstract}

In the ciruit X and Y are digital inputs, Z is digital output.The equivalent circuit is the logic implementation of XOR Gate.
\begin{figure}[h]
    \centering
    \includegraphics[scale=0.2]{xor.png}
    \caption{Z=X!Y+!XY}
    %\caption{Circuit}
    \label{fig:circuit}
\end{figure}
\section{\textbf{Components}}
\input{components}
\begin{table}[!h]
\centering
\caption{}
\label{table:7447_disp}
\end{table}
\begin{figure}
The figure given below is the pin diagram of Seven Segment Display\\
    \centering
    \includegraphics{seven.png}
    \caption{Seven segment display}
    \label{fig:my_label}
\end{figure}
\begin{figure}
The table given below is the connections between 7447 BCD Decoder and Seven Segment Display\\
    \centering
    \includegraphics[scale=0.5]{sevenseg.png}
    \caption{}
    \label{fig:my_label}
\end{figure}
\begin{figure}
The diagram below shows the pin diagram of 7447 BCD Decoder.The output pins of 7447 is connected to Seven Segment Display using fig 3.
    \centering
    \includegraphics[scale=0.5]{ic7447.png}
    \caption{}
    \label{fig:my_label}
\end{figure}

\section{Procedure}
1. connect the circuit using 7447 BCD-Seven segment display decoder and Arduino\\
2.  connect the seven segment pins to 7447 using fig 3.\\
3. connect the input pins X,Y to 0 and 1.\\
4. connect the output pins of 7447 to Gnd except pin A to D2.\\
5. change the input pins according to XOR logic and verify the output.\\ 
6. Verify the XOR operation in Arduino using the following code and making
pin connections according to fig 2,3.

\textbf{Observe the circuit and verify the program by executing the link provided below.}\\
\begin{center}
\fbox{\parbox{8.5cm}{\url{https://github.com/sindhu023/FWC/}}}
\end{center}
\end{document}
