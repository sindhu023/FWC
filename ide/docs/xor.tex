\documentclass[journal,12pt,twocolumn]{IEEEtran}
%
\usepackage{setspace}
\usepackage{textcomp}
\usepackage{gensymb}
\usepackage{xcolor}
\usepackage{caption}
%\usepackage{subcaption}
%\doublespacing
\singlespacing

\usepackage{graphicx}
\graphicspath{{./images/}}
\usepackage[colorlinks=true, urlcolor=blue, linkcolor=black]{hyperref}
%\usepackage[parfill]{parskip}
%\usepackage{amssymb}
%\usepackage{relsize}
\usepackage[cmex10]{amsmath}
\usepackage{mathtools}
%\usepackage{amsthm}
%\interdisplaylinepenalty=2500
%\savesymbol{iint}
%\usepackage{txfonts}
%\restoresymbol{TXF}{iint}
%\usepackage{wasysym}
\usepackage{amsthm}
\usepackage{mathrsfs}
\usepackage{txfonts}
\usepackage{stfloats}
\usepackage{cite}
\usepackage{cases}
\usepackage{subfig}
%\usepackage{xtab}
\usepackage{hyperref}
\usepackage{longtable}
\usepackage{multirow}
%\usepackage{algorithm}
%\usepackage{algpseudocode}
\usepackage{enumitem}
\usepackage{mathtools}
%\usepackage{eenrc}
%\usepackage[framemethod=tikz]{mdframed}
%\usepackage{hyperref}
\usepackage{listings}
    \usepackage[latin1]{inputenc}                                 %%
    \usepackage{color}                                            %%
    \usepackage{array}                                            %%
    \usepackage{longtable}                                        %%
    \usepackage{calc}                                             %%
    \usepackage{multirow}                                         %%
    \usepackage{hhline}                                           %%
    \usepackage{ifthen}                                           %%
  %optionally (for landscape tables embedded in another document): %%
    \usepackage{lscape}     
\usepackage{tikz}
\usepackage{circuitikz}
\usepackage{karnaugh-map}
\usepackage{pgf}

\usepackage{url}
\def\UrlBreaks{\do\/\do-}



%\usepackage{stmaryrd}


%\usepackage{wasysym}
%\newcounter{MYtempeqncnt}
\DeclareMathOperator*{\Res}{Res}
%\renewcommand{\baselinestretch}{2}
\renewcommand\thesection{\arabic{section}}
\renewcommand\thesubsection{\thesection.\arabic{subsection}}
\renewcommand\thesubsubsection{\thesubsection.\arabic{subsubsection}}

\renewcommand\thesectiondis{\arabic{section}}
\renewcommand\thesubsectiondis{\thesectiondis.\arabic{subsection}}
\renewcommand\thesubsubsectiondis{\thesubsectiondis.\arabic{subsubsection}}



%\surroundwithmdframed[width=\columnwidth]{lstlisting}
\def\inputGnumericTable{}                                 %%
\lstset{
%language=C,
frame=single, 
breaklines=true,
columns=fullflexible
}
 

\begin{document}
%

\theoremstyle{definition}
\newtheorem{theorem}{Theorem}[section]
\newtheorem{problem}{Problem}
\newtheorem{proposition}{Proposition}[section]
\newtheorem{lemma}{Lemma}[section]
\newtheorem{corollary}[theorem]{Corollary}
\newtheorem{example}{Example}[section]
\newtheorem{definition}{Definition}[section]
%\newtheorem{algorithm}{Algorithm}[section]
%\newtheorem{cor}{Corollary}
\newcommand{\BEQA}{\begin{eqnarray}}
\newcommand{\EEQA}{\end{eqnarray}}
\newcommand{\define}{\stackrel{\triangle}{=}}
\vspace{2cm}
\title{ 
XOR logic through Arduino
}

\author{B.Sai Sindhu}


\maketitle
\tableofcontents
%
%\newpage
\section{Abstract}

\begin{abstract}
This manual shows how to implement XOR logic through Arduino.
\end{abstract}

In the ciruit X and Y are digital inputs, Z is digital output.The equivalent circuit is the logic implementation of XOR Gate.
\begin{figure}[h]
    \centering
    \includegraphics[scale=0.2]{xor.png}
    \caption{Z=X!Y+!XY}
    %\caption{Circuit}
    \label{fig:circuit}
\end{figure}
\section{\textbf{Components}}
\input{components}
\begin{table}[!h]
\centering
\caption{}
\label{table:7447_disp}
\end{table}

\begin{figure}[!h]
1.The figure given below is the pin diagram of Seven Segment Display.\\
\begin{center}
\resizebox {0.4\columnwidth} {!} {
\input{sevenseg.tex}
}
\end{center}
\caption{}
\label{fig:sevenseg}
\end{figure}

\begin{table}[!h]
2.The table given below is the connections between 7447 BCD Decoder and Seven Segment Display\\
\centering
\input{7447_disp.tex}
\caption{}
\label{table:7447_disp}
\end{table}

\begin{figure}[!h]
3.The diagram below shows the pin diagram of 7447 BCD Decoder.The output pins of 7447 is connected to Seven Segment Display using Table 2.
\begin{center}
\resizebox {1.2\columnwidth} {!} {
\input{7447.tex}
}
\end{center}
\caption{}
\label{fig:7447}
\end{figure}

\section{Procedure}
1. connect the circuit using 7447 BCD-Seven segment display decoder and Arduino.\\
2. connect the seven segment pins to 7447 using Table 2.\\
3. connect the pin A of 7447 to D2 of Arduino and remaining pins B,C and D to GND. \\
4. connect the pins D5,D6 to 1 and 0.Change the pins simultaneously to verify the XOR truth table.
5. Verify the XOR operation in arduino using the following code and making
pin connections according to fig 2,Table 2 .

\textbf{Observe the circuit and verify the program by executing the link provided below.}\\
\begin{center}
\fbox{\parbox{8.5cm}{\url{https://github.com/sindhu023/FWC/}}}
\end{center}
\end{document}
